\newpage

\pagenumbering{Roman}
\setcounter{page}{1}

\section*{写在前面}
\addcontentsline{toc}{section}{写在前面}

一直以来都有这样的愿望:无论学习什么知识,总是希望可以快速准确地找到对应的有价值资源进行学习。我相信我们每个人都梦寐以求。然而,越来越多的学科,尤其是我目前从事的计算机科学、人工智能领域,当下正在飞速地发展着。太多的新知识都难以事半功倍地找到快速入手的教程。庄子曰:\textit{“吾生也有涯,而知也无涯。以有涯随无涯,殆已。”}

我只是迁移学习领域一个很普通的博士生,也同样经历了由“一问三不知”到“稍稍理解”的艰难过程。我在2016年初入门迁移学习之时,迁移学习这个概念还未曾像今天一样炙手可热。当时所能找到的学习资源只有两种:别人已发表的论文和已做过的演讲。这些还是不够简单、不够直观。我需要从如此众多的材料中不断归纳,才能站在博士研究的那个圈子的边缘,以便将来可以做出一点点贡献,往圆圈外突破一点点。

相信不只是我,任何一个刚刚入门的学习者都会经历此过程。

\textit{“沉舟侧畔千帆过,病树前头万木春。”}

已所不欲,勿施于人。正是因为我在初学之时也经历过如此沮丧的时期,我才在Github上对迁移学习进行了整理归纳,在知乎网上以“\textit{王晋东不在家}”为名分享自己对于迁移学习和机器学习的理解和教训、在线上线下与大家讨论相关的问题。很欣慰的是,这些免费开放的资源或多或少地,帮助到了一些初学者,使他们更快速地步入迁移学习之门。

但这些还是不太够。Github上的资源模式已经固定,目前主要是进行日常更新,不断加入新的论文和代码。目前还是缺乏一个人人都能上手的初学者教程。也只一次,有读者提问有没有相关的入门教程,能真正从0到1帮助初学者进行入门。

最近,南京大学博士(现任旷视科技南京研究院负责人)魏秀参学长写了一本《解析卷积神经网络—深度学习实践手册》,给很多深度学习的初学者提供了帮助。受他的启发,我也决定将自己在迁移学习领域的一些学习心得体会整理成一本手册,免费进行分享。希望能借此方式,帮助更多的初学者。\textit{我们不谈风月,只谈干货。}

我不是大佬,我也是迁移学习路上的一名小学生。迁移学习领域比我做的好的同龄人太多了。因此,不敢谈什么\textit{指导}。所有的目的都仅为\textit{分享}。

本手册在互联网上免费开放。随着作者理解的深入(以及其他有意者的增补),本手册肯定会不断修改、越来越好。因此,我打算效仿软件的开发、采取版本更新的方式进行管理。

希望未来可以有更多的有志之士加入,让我们的教程日渐丰富。

\newpage

\section*{致谢}
\addcontentsline{toc}{section}{致谢}

本手册编写过程中得到了许多人的帮助。在此对他们表示感谢。

感谢我的导师、中国科学院计算技术研究所的陈益强研究员。是他一直以来保持着对我的信心,相信我能做出好的研究成果,不断鼓励我,经常与我讨论以明确问题,才有了今天的我。陈老师给我提供了优良的实验环境。我一定会更加努力地科研,做出更多更好的研究成果。

感谢香港科技大学计算机系的杨强教授。杨教授作为迁移学习领域国际泰斗,经常不厌其烦地回答我一些研究上的问题。能够得到杨教授的指导,是我的幸运。希望我能在杨教授带领下,做出更踏实的研究成果。

感谢新加坡南洋理工大学的于涵老师。作为我论文的共同作者,于老师认真的写作态度、对论文的把控能力是我一直学习的榜样。于老师还经常鼓励我,希望可以和于老师有着更多合作,发表更好的文章。

感谢清华大学龙明盛助理教授。龙老师在迁移学习领域发表了众多高质量的研究成果,是我入门时学习的榜样。龙老师还经常对我的研究给予指导。希望有机会可以真正和龙老师合作。

感谢美国伊利诺伊大学芝加哥分校的Philip S. Yu教授对我的指导和鼓励。

感谢新加坡A*STAR的郝书吉老师。我博士生涯的发表的第一篇论文是和郝老师合作完成的。正是有了第一篇论文被发表,才增强了我的自信,在接下来的研究中放平心态。

感谢我的好基友、西安电子科技大学博士生段然同学和我的同病相怜,让我们可以一起吐槽读博生活。

感谢我的室友沈建飞、以及实验室同学的支持。

感谢我的知乎粉丝和所有交流过迁移学习的学者对我的支持。

最后感谢我的女友和父母对我的支持。

本手册中出现的插图,绝大多数来源于相应论文的配图。感谢这些作者做出的优秀的研究成果。希望我能早日作出可以比肩的研究。

\newpage
\section*{说明}
\addcontentsline{toc}{section}{手册说明}
本手册的编写目的是帮助迁移学习领域的初学者快速进行入门。我们尽可能绕开那些非常理论的概念,只讲经验方法。我们还配有多方面的代码、数据、论文资料,最大限度地方便初学者。

本手册的方法部分,关注点是近年来持续走热的领域自适应(Domain Adaptation)问题。迁移学习还有其他众多的研究领域。由于作者研究兴趣所在和能力所限,对其他部分的研究只是粗略介绍。非常欢迎从事其他领域研究的读者提供内容。

本手册的每一章节都是\textit{自包含}的,因此,初学者不必从头开始阅读每一部分。直接阅读自己需要的或者自己感兴趣的部分即可。本手册每一章节的信息如下:

第1章介绍了迁移学习的概念,重点解决什么是迁移学习、为什么要进行迁移学习这两个问题。

第2章介绍了迁移学习的研究领域。

第3章介绍了迁移学习的应用领域。

第4章是迁移学习领域的一些基本知识,包括问题定义,域和任务的表示,以及迁移学习的总体思路。特别地,我们提供了较为全面的度量准则介绍。度量准则是迁移学习领域重要的工具。

第5章简要介绍了迁移学习的四种基本方法,即基于样本迁移、基于特征迁移、基于模型迁移、基于关系迁移。

第6章到第8章,介绍了领域自适应的3大类基本的方法,分别是:数据分布自适应法、特征选择法、子空间学习法。

第9章重点介绍了目前主流的深度迁移学习方法。

第10章提供了简单的上手实践教程。

第11章对迁移学习进行了展望,提出了未来几个可能的研究方向。

第12章是对全手册的总结。

第13章是附录,提供了迁移学习领域相关的学习资源,以供读者参考。

\textit{\\由于作者水平有限,不足和错误之处,敬请不吝批评指正。}

\textbf{手册的相关资源:}

我们在Github上持续维护了迁移学习的资源仓库,包括论文、代码、文档、比赛等,请读者持续关注:\url{https://github.com/jindongwang/transferlearning},配合本手册更香哦!

网站(内含勘误表):\url{http://t.cn/RmasEFe}

开发维护地址: \url{http://github.com/jindongwang/transferlearning-tutorial}

作者的联系方式:

\textit{邮箱}: {\ttfamily jindongwang@outlook.com},\textit{知乎}:{\ttfamily 王晋东不在家}。

\textit{微博}:{\ttfamily 秦汉日记},\textit{个人网站}:\url{http://jd92.wang}。
